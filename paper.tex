\documentclass[oneside, 12pt]{amsart}
\usepackage{amscd, amsmath, amssymb, amsthm, amsfonts, amstext, geometry, verbatim, enumitem, graphicx, mathtools, xfrac, tikz-cd, microtype, nameref, thmtools}
\usepackage[breaklinks=true]{hyperref}
\usepackage[capitalize]{cleveref}

\definecolor{darkblue}{rgb}{0.0, 0.0, 0.6}

\hypersetup{
  pdfauthor={Sergey Sinchuk, Andrei Smolensky},
  pdftitle={Decompositions of congruence subgroups of Chevalley groups},  
  colorlinks=true,
  urlcolor=darkblue,
  linkcolor=darkblue,
  citecolor=darkblue}

\usepackage[hyperref=true, backend=bibtex, citestyle=numeric-comp, sortlocale=en_US, url=false, doi=false, eprint=true, maxbibnames=4]{biblatex}            
\addbibresource{paper.bib}
\renewbibmacro*{volume+number+eid}{\ifentrytype{article}{\- \iffieldundef{volume}{}{Vol.~\printfield{volume},}\iffieldundef{number}{}{ No.~\printfield{number},}}}
\renewbibmacro{in:}{\ifentrytype{article}{}{\printtext{\bibstring{in}\intitlepunct}}}
\newbibmacro{string+doi}[1]{\iffieldundef{doi}{\iffieldundef{url}{#1}{\href{\thefield{url}}{#1}}}{\href{http://dx.doi.org/\thefield{doi}}{#1}}}
\DeclareFieldFormat[article, inproceedings, inbook, thesis]{title}{\usebibmacro{string+doi}{\mkbibquote{#1}}}
\renewcommand*{\bibfont}{\footnotesize}

\oddsidemargin 5mm
\marginparwidth 5mm
\topmargin 0mm
\textheight 225mm
\textwidth 165mm
\headheight 0mm
\headsep 10mm
\footskip 5mm

\newlist{thmlist}{enumerate}{1} \setlist[thmlist]{label={\rm(\arabic{thmlisti})}, ref=\thethm.(\roman{thmlisti}),noitemsep} \Crefname{thmlisti}{Theorem}{Theorems}
\newlist{lemlist}{enumerate}{1} \setlist[lemlist]{label={\rm(\arabic{lemlisti})}, ref=\thelemma.(\roman{lemlisti}),noitemsep} \Crefname{lemlisti}{Lemma}{Lemmas}

\theoremstyle{plain}

\newtheorem{thm}{Theorem}
\Crefname{thm}{Theorem}{Theorems}
\numberwithin{equation}{section}

\newtheorem{lemma}{Lemma}
\numberwithin{lemma}{section}
\Crefname{lemma}{Lemma}{Lemmas}

\newtheorem{cor}[lemma]{Corollary}
\Crefname{cor}{Corollary}{Corollaries}

\newtheorem{prop}[lemma]{Proposition}
\Crefname{prop}{Proposition}{Propositions}

\newtheorem*{thm*}{Theorem}
\newtheorem*{lemma*}{Lemma}

\theoremstyle{definition}

\newtheorem{dfn}[lemma]{Definition}
\Crefname{dfn}{Definition}{Definitions}
\newtheorem{example}[lemma]{Example}
\Crefname{example}{Example}{Examples}

\theoremstyle{remark}

\newtheorem{rem}[lemma]{Remark}
\Crefname{rem}{Remark}{Remarks}

\DeclareMathOperator{\K}{K}
\DeclareMathOperator{\SK}{SK}
\DeclareMathOperator{\G}{G}
\DeclareMathOperator{\GL}{GL}
\DeclareMathOperator{\SL}{SL}
\DeclareMathOperator{\Sp}{Sp}
\DeclareMathOperator{\SO}{SO}
\DeclareMathOperator{\St}{St}
\DeclareMathOperator{\E}{E}
\DeclareMathOperator{\EP}{EP}
\DeclareMathOperator{\Par}{P}
\DeclareMathOperator{\Hom}{Hom}
\DeclareMathOperator{\B}{B}
\DeclareMathOperator{\Hh}{H}
\DeclareMathOperator{\U}{U}
\DeclareMathOperator{\Z}{Z}
\DeclareMathOperator{\M}{M}
\DeclareMathOperator{\SR}{SR}
\DeclareMathOperator{\sr}{sr}
\DeclareMathOperator{\asr}{asr}
\DeclareMathOperator{\shape}{shape}
\DeclareMathOperator{\Rad}{Rad}
\DeclareMathOperator{\Max}{Max}
\DeclareMathOperator{\Spec}{Spec}
\DeclareMathOperator{\Spin}{Spin}
\DeclareMathOperator{\Epin}{Epin}
\DeclareMathOperator{\Stab}{Stab}
\DeclareMathOperator{\ASR}{ASR}
\DeclareMathOperator{\Ums}{Ums}
\DeclareMathOperator{\Umd}{Umd}
\DeclareMathOperator{\rk}{rk}
\newcommand{\rA}{\mathsf{A}}
\newcommand{\rB}{\mathsf{B}}
\newcommand{\rC}{\mathsf{C}} 
\newcommand{\rD}{\mathsf{D}} 
\newcommand{\rE}{\mathsf{E}}
\newcommand{\rF}{\mathsf{F}}
\newcommand{\rG}{\mathsf{G}}

\makeatletter
\newcommand{\indexbox}[1]{\text{\fboxsep=.1em\fbox{\m@th$\displaystyle#1$}}}
\makeatother

\def\ssub#1{\mathchoice
   {_{\lower2pt\hbox{$\scriptstyle #1$}}}
   {_{\lower2pt\hbox{$\scriptstyle #1$}}}
   {_{\lower1.5pt\hbox{$\scriptscriptstyle #1$}}}
   {_{\!\lower1.5pt\hbox{$\scriptscriptstyle #1$}}}}
   
\title [Subsystem factorizations] {Subsystem factorizations}
\keywords { {\em Mathematical Subject Classification (2010):} 20G40, 20G35, 19B14}
\author{Sergey Sinchuk}
\author{Andrei Smolensky}
\email{sinchukss {\it at} yandex.ru; andrei.smolensky {\it at} gmail.com}
\thanks{Authors of the present paper acknowledge the financial support from Russian Science Foundation grant 14-11-00297}

\date {\today}

\begin{document}

\begin{abstract} To be written. \end{abstract}

\maketitle

\section {Introduction}\label{intro}
 
\subsection{Acknowledgements}
The authors of the present paper acknowledges financial support from Russian Science Foundation grant 14-11-00297.

\section {Preliminaries}\label{prelim}

\section{Computations}
\begin{lemma}\label{lemma:e6-d5}
Let $R$ be a commutative ring with $\asr(R)\leqslant 4$. Then $\G(\rE_6, R)$ equals the product of 5 of its subgroups isomorphic to $\G(\rD_5, R)$.
\end{lemma}
\begin{proof}
By Dennis---Vaserstein decomposition one has
\[ \G(\rE_6) = \G(\Delta_1)\U(\Sigma_1)\U(\Sigma_1^-\cap\Sigma_6^-)\U(\Sigma_6)\G(\Delta_6). \]
We divide $\Sigma_1$ into two parts: $\Sigma_1=(\Sigma_1\cap\Delta_6)\cup(\Sigma_1\cap\Sigma_6)$. The set $\Sigma_1\cap\Sigma_6$ lies inside the $\rD_5$-subsystem spanned by $\alpha_3$, $\alpha_4$, $\alpha_5$, $\alpha_2$ and $\alpha_{\max}$. Thus we can rewrite
\begin{multline*}
\G(\rE_6) = \G(\Delta_1) \U(\Sigma_1) \U(\Sigma_1^-\cap\Sigma_6^-) \U(\Sigma_6) \G(\Delta_6) = \\
= \G(\Delta_1) \U(\Sigma_1\cap\Delta_6) \U(\Sigma_1\cap\Sigma_6) \U(\Sigma_1^-\cap\Sigma_6^-) \U(\Sigma_1\cap\Sigma_6) \U(\Sigma_6\cap\Delta_1) \G(\Delta_6) \subseteq \\
\subseteq \G(\Delta_1) \G(\Delta_6) \G(\langle\alpha_3, \alpha_4, \alpha_5, \alpha_2, \alpha_{\max}\rangle) \G(\Delta_1) \G(\Delta_6).\qedhere
\end{multline*}
\end{proof}

\begin{lemma}
Let $R$ be a commutative ring with $\sr(R)\leqslant3$. Then $\G(\rE_7, R)$ equals the product of 16 of its subgroups isomorphic to $\G(\rD_6, R)$ or $\G(\rA_7, R)$.
\end{lemma}
\begin{proof}
We start with the following Dennis---Vaserstein decomposition (yet unpublished):
\[ \G(\rE_7) = \G(\Delta_1) \U(\Sigma_1) \U(\Sigma_1^-\cap\Sigma_2^-) \U(\Sigma_2) \G(\Delta_2). \]
We divide the unipotents sets of roots as follows
\begin{align*}
& \U(\Sigma_1) = \U(\Sigma_1\cap\Delta_2) \U(\Sigma_1\cap\Sigma_2), \\
& \U(\Sigma_2) = \U(\Sigma_2\cap\Sigma_1) \U(\Sigma_2\cap\Delta_1), \\
& \U(\Sigma_1\cap\Sigma_2) = \U(\Sigma_1\cap\Sigma_2^{=1}) \U(\Sigma_1\cap\Sigma_2^{=2}).
\end{align*}
The set of roots $\Sigma_1\cap\Sigma_2^{=2}$ lies inside the obvious $\rA_7$-subsystem. The set $\Sigma_1\cap\Sigma_2^{=1}$ spans the $\rE_6$-subsystem. Damn!

Thus
\begin{align*}
\G(\rE_7) ={} & \G(\Delta_1) \U(\Sigma_1\cap\Delta_2) \U(\Sigma_1\cap\Sigma_2^{=2}) \cdot{} \\
& \cdot \U(\Sigma_1\cap\Sigma_2^{=1}) \U(\Sigma_1^-\cap\Sigma^{=-1}) \cdot{} \\
& \cdot \U(\Sigma_1^-\cap\Sigma_2^{=-2}) \U(\Sigma_1\cap\Sigma_2^{=2}) \cdot{} \\
& \cdot \U(\Sigma_1\cap\Sigma_2^{=1}) \U(\Sigma_2\cap\Delta_1) \G(\Delta_2) = \\
={} & \G(\Delta_1) \G(\Delta_2) \G(\rA_7) \G(\rE_6) \G(\rA_7) \U(\Sigma_1\cap\Sigma_2^{=1}) \G(\Delta_1) \G(\Delta_2).
\end{align*}
By \cref{lemma:e6-d5} one decomposes each $\rE_6$-factor into the product of 5 $\rD_5$-subgroups, so in total one gets the product of 16 classical factors.
\end{proof}
Perhaps we should further divide $\Sigma_1\cap\Sigma_2^{=1}$ to obtain a shorter decomposition.
\printbibliography

\end{document}
